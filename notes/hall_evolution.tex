\documentclass[letterpaper,10pt]{article}
\usepackage[utf8]{inputenc}
\usepackage{amsmath}
\usepackage{amssymb}
\usepackage{graphicx}
\usepackage{listings}
\usepackage{float}
%\bibliographystyle{plainnat}
\newcommand{\dd}{\,\mathrm{d}\,}
\newcommand{\pp}{\partial}
\newcommand{\co}{\mathrm{.}}
\newcommand{\cosec}{\mathrm{cosec}}
\newcommand{\cotan}{\mathrm{cotan}}
\newcommand{\D}{\displaystyle}

\renewcommand{\vec}[1]{\boldsymbol#1}
\newcommand{\bhat}[1]{\hat{\boldsymbol#1}}

\setlength\topmargin{0in}
\setlength\headheight{0in}
\setlength\headsep{0in}
\setlength\textheight{8.5in}
\setlength\textwidth{6.5in}
\setlength\oddsidemargin{0in}
\setlength\evensidemargin{0in}
\setlength\parindent{0.25in}
\setlength\parskip{0.25in}

%opening
\title{Hall Evolution}
\author{Pablo Marchant Campos}
\begin{document}
\maketitle
I will consider a magnetic field that evolves according to the eq.
\begin{eqnarray}
\frac{\dd \vec{B}}{\dd t}=-\nabla\times\left(\frac{c}{4\pi ne}[\nabla\times\vec{B}]\times\vec{B}+\eta\nabla\times\vec{B}\right),\label{intro::timeeq}
\end{eqnarray}
where $n$ is the electron density and $\eta\equiv c^2/(4\pi\sigma)$ (where $\sigma$ is the electrical conductivity) is the magnetic diffusivity. For simplicity i will consider both of these quantities to be radial functions, and time independent.

Restricting this to an axisymmetric field, $\vec{B}$ can be written in terms of two scalar functions as
\begin{eqnarray}
\vec{B}=\nabla\alpha(r,\theta)\times\nabla\phi+\beta\nabla\phi,
\end{eqnarray}
where $\phi$ is the azimuth in spherical coordinates. Thus, $\alpha(r,\theta)$ describes the poloidal field and $\beta(r,\theta)$ the toroidal field. The curl of $\vec{B}$ can be expressed in terms of these functions,
\begin{eqnarray}
\nabla\times\vec{B}=-\Delta\alpha\nabla\phi+\nabla \beta\times\nabla\phi.,\quad \Delta\equiv=\varpi^2\nabla\cdot(\varpi^{-2}\nabla)=\pp_r^2+\frac{\sin\theta}{r^2}\pp_\theta\left(\frac{\pp_\theta}{\sin\theta}\right)
\end{eqnarray}
Using this, and defining $\chi\equiv c/(4\pi en\varpi^2)$, where $\varpi\equiv r\sin\theta$ is the radial coordinate, eq. (\ref{intro::timeeq}) gives
\begin{eqnarray}
\nabla\left(\frac{\pp \alpha}{\pp t}\right)\times\nabla\phi+\frac{\pp\beta}{\pp t}\nabla\phi=
\nabla\times\left(\varpi^2\chi[\Delta\alpha\nabla\phi-\nabla\beta\times\nabla\phi]\times[\nabla\alpha\times\nabla\phi+\beta\nabla\phi]-\eta[-\Delta\alpha\nabla\phi+\nabla\beta\times\nabla\phi]\right).
\end{eqnarray}
After expanding the cross product inside the curl, each term in this equation is either poloidal or toroidal, and taking this into account, i write two equations which together are equivalent to the previous one,
\begin{eqnarray}
\begin{aligned}
\nabla\left(\frac{\pp \alpha}{\pp t}\right)\times\nabla\phi=&
\nabla\times\left(\varpi^2\chi[\nabla\beta\times\nabla\phi]\times[\nabla\alpha\times\nabla\phi]+\eta\Delta\alpha\nabla\phi\right),\\
\frac{\pp\beta}{\pp t}\nabla\phi=&
\nabla\times\left(\varpi^2\chi\Delta\alpha\nabla\phi\times[\nabla\alpha\times\nabla\phi]+\varpi^2\chi\beta\nabla\phi\times[\nabla\beta\times\nabla\phi]-\eta\nabla\beta\times\nabla\phi\right).
\end{aligned}
\end{eqnarray}
With a couple of changes, these equations can be transformed into equations for the scalar functions $\alpha$ and $\beta$,
\begin{eqnarray}
\begin{aligned}
\frac{\pp\alpha}{\pp t}=&\varpi^2\chi[\nabla\alpha\times\nabla\beta]\cdot\nabla\phi+\eta\Delta\alpha\\
\frac{\pp \beta}{\pp t}=&\varpi^2(\nabla[\chi\Delta\alpha]\times\nabla\alpha+\beta\nabla\chi\times\nabla\beta)\cdot\nabla\phi+\varpi^2\nabla\cdot\left(\frac{\eta\nabla\beta}{\varpi^2}\right).
\end{aligned}
\end{eqnarray}
I will consider an adimensional form of this equation, in which each quantity is expressed in terms of characteristic values $B_0$ for the magnetic field, $\chi_0$ for $\chi$ and the radius of the star $R$ for the radial coordinate $r$. These characteristic values define two timescales
\begin{eqnarray}
t_h\equiv(\chi_0B_0)^{-1},\quad t_d\equiv\frac{R^2}{\eta}
\end{eqnarray}
and if I consider my unit of time to be $t_h$, the equations for $\alpha$ and $\beta$ (considering all the quantities in their adimensional versions) are given by
\begin{eqnarray}
\begin{aligned}
\frac{\pp\alpha}{\pp t}=&\varpi^2\chi[\nabla\alpha\times\nabla\beta]\cdot\nabla\phi+\left(\frac{t_h}{t_d}\right)\eta\Delta\alpha\\
\frac{\pp \beta}{\pp t}=&\varpi^2(\nabla[\chi\Delta\alpha]\times\nabla\alpha+\beta\nabla\chi\times\nabla\beta)\cdot\nabla\phi+\left(\frac{t_h}{t_d}\right)\varpi^2\nabla\cdot\left(\frac{\eta\nabla\beta}{\varpi^2}\right).
\end{aligned}\label{intro::eqs}
\end{eqnarray}

\section{Toroidal Field}
To begin, I will consider a toroidal field. It is clear from eq. (\ref{intro::eqs}) that a purely toroidal field will remain toroidal. Although in this section I take $\alpha=0$, I will show first that the eq. for $\beta$ in (\ref{intro::eqs}) can be written as a conservative eq.,
\begin{eqnarray}
\frac{1}{\varpi^2}\frac{\pp\beta}{\pp t}=&\nabla\times(\chi\Delta\alpha\nabla\alpha+\chi\beta\nabla\beta)\cdot\phi+\left(\frac{t_h}{t_d}\right)\nabla\cdot\left(\frac{\eta\nabla\beta}{\varpi^2}\right)\\
\frac{1}{\varpi^2}\frac{\pp\beta}{\pp t}=&\nabla\cdot(\nabla\phi\times[\chi\Delta\alpha\nabla\alpha+\chi\beta\nabla\beta])+\left(\frac{t_h}{t_d}\right)\nabla\cdot\left(\frac{\eta\nabla\beta}{\varpi^2}\right),\label{tor::cons}
\end{eqnarray}
so I expect the quantity
\begin{eqnarray}
I_1=\int_V\frac{\beta}{\varpi^2}\dd V
\end{eqnarray}
to remain constant unless the "fluxes" in eq. (\ref{tor::cons}) are not zero in the boundaries. Since $\beta/\varpi=B_{tor}$, where $B_{tor}$ is the toroidal component of the magnetic field, $I_1$ can also be written as
\begin{eqnarray}
I_1=2\pi\int_0^\pi\int_0^R B_{tor}r\dd r\dd\theta,
\end{eqnarray}
which is simply the magnetic flux that crosses a half cut of the star that crosses the symmetry axis. So, when considering the evolution of the system under axial symmetry, the toroidal flux is a conserved quantity.

Now, let us consider the case $\alpha=0$. Expressing eq. (\ref{tor::cons}) in terms of spherical coordinates gives
\begin{eqnarray}
\frac{\pp}{\pp t}\left(\frac{\beta}{\varpi^2}\right)=\frac{1}{r^2} \frac{\pp}{\pp r}\left(\frac{\chi\beta}{\sin\theta}\frac{\pp \beta}{\pp\theta}+\left(\frac{t_h}{t_d}\right)\frac{\eta r^2}{\varpi^2}\frac{\pp\beta}{\pp r}\right)+\frac{1}{\varpi}\frac{\pp}{\pp\theta}\left(-\frac{\chi\beta}{r}\frac{\pp\beta}{\pp r}+\left(\frac{t_h}{t_d}\right)\frac{\sin\theta \eta}{r\varpi^2}\frac{\pp\beta}{\pp\theta}\right),
\end{eqnarray}
which after multiplication by $r^2\sin\theta$ gives an equation for which a conservative code can be readily constructed,
\begin{eqnarray}
\frac{\pp}{\pp t}\left(\frac{\beta}{\sin\theta}\right)=\frac{\pp}{\pp r}\left(\chi\beta\frac{\pp \beta}{\pp\theta}+\left(\frac{t_h}{t_d}\right)\frac{\eta}{\sin\theta}\frac{\pp\beta}{\pp r}\right)+\frac{\pp}{\pp\theta}\left(-\chi\beta\frac{\pp\beta}{\pp r}+\left(\frac{t_h}{t_d}\right)\frac{\eta}{r^2\sin\theta}\frac{\pp\beta}{\pp\theta}\right).
\end{eqnarray}
In order to perform the simulation, I will consider a discretization on a grid with spacing $\Delta r$ in the radial coordinate and $\Delta \theta$ in the angular coordinate, and a timestep $\Delta t$. In this way, I will have $\beta(r,\theta,t)\rightarrow \beta_{i,j}^n$, where the indexes $i,j,n$ represent steps in radius, angle and time. The conservative scheme will then be
\begin{eqnarray}
\begin{aligned}
\frac{\beta_{i,j}^{n+1}-\beta_{i,j}^{n}}{\Delta t\sin (j\Delta)\theta}=&\D\frac{1}{\Delta r}\left[\left(\chi\beta\frac{\pp \beta}{\pp\theta}+\left(\frac{t_h}{t_d}\right)\frac{\eta}{\sin\theta}\frac{\pp\beta}{\pp r}\right)_{i+1/2,j}-\left(\chi\beta\frac{\pp \beta}{\pp\theta}+\left(\frac{t_h}{t_d}\right)\frac{\eta}{\sin\theta}\frac{\pp\beta}{\pp r}\right)_{i-1/2,j}\right]\\
&\D+\frac{1}{\Delta\theta}\left[\left(-\chi\beta\frac{\pp\beta}{\pp r}+\left(\frac{t_h}{t_d}\right)\frac{\eta}{r^2\sin\theta}\frac{\pp\beta}{\pp\theta}\right)_{i,j+1/2}-\left(-\chi\beta\frac{\pp\beta}{\pp r}+\left(\frac{t_h}{t_d}\right)\frac{\eta}{r^2\sin\theta}\frac{\pp\beta}{\pp\theta}\right)_{i,j-1/2}\right]
\end{aligned}
\end{eqnarray}
\begin{eqnarray}
\begin{aligned}
\frac{\beta_{i,j}^{n+1}-\beta_{i,j}^{n}}{\Delta t\sin (j\Delta)\theta}=\D\frac{1}{\Delta r}\left[\frac{1}{8\Delta\theta}(\beta_{i+1,j}+\beta_{i,j})(\beta_{i,j+1}+\beta_{i+1,j+1}-\beta_{i,j-1}-\beta_{i+1,j-1})\chi([i+1/2]\Delta r,j\Delta\theta)\right.\\
\left.-\frac{1}{8\Delta\theta}(\beta_{i,j}+\beta_{i-1,j})(\beta_{i,j+1}+\beta_{i-1,j+1}-\beta_{i,j-1}-\beta_{i-1,j-1})\chi([i+1/2]\Delta r,j\Delta\theta)\right.\\
+\left.\left(\frac{t_h}{t_d}\right)\frac{1}{\Delta r\sin(j\Delta\theta)}\eta([i+1/2]\Delta r,j\Delta\theta)(\beta_{i+1,j}-\beta_{i,j})\right.\\
\left.-\left(\frac{t_h}{t_d}\right)\frac{1}{\Delta r\sin(j\Delta\theta)}\eta([i-1/2]\Delta r,j\Delta\theta)(\beta_{i,j}-\beta_{i-1,j})\right]\\
+\frac{1}{\Delta \theta}\left[\frac{1}{8\Delta r}(\beta_{i,j+1}+\beta_{i,j})(\beta_{i+1,j+1}+\beta_{i+1,j}-\beta_{i-1,j}-\beta_{i-1,j+1})\chi(i\Delta r,[j+1/2]\Delta\theta)\right.\\
\left.-\frac{1}{8\Delta r}(\beta_{i,j}+\beta_{i,j-1})(\beta_{i+1,j-1}+\beta_{i+1,j}-\beta_{i-1,j}-\beta_{i-1,j-1})\chi(i\Delta r,[j-1/2]\Delta\theta)\right.\\
+\left.\left(\frac{t_h}{t_d}\right)\frac{1}{\Delta \theta(i\Delta r)^2}\frac{\eta(i\Delta r,[j+1/2]\Delta\theta)(\beta_{i,j+1}-\beta_{i,j})}{\sin([j+1/2]\Delta\theta)}\right.\\
\left.-\left(\frac{t_h}{t_d}\right)\frac{1}{\Delta \theta(i\Delta r)^2}\frac{\eta(i\Delta r,[j-1/2]\Delta\theta)(\beta_{i,j}-\beta_{i,j-1})}{\sin([j-1/2]\Delta\theta)}\right].
\end{aligned}
\end{eqnarray}
This scheme is implemented in the code as_toroidal.cpp
\end{document}















